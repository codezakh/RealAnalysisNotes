\documentclass{article}
\usepackage{amssymb}
\usepackage{amsmath}

\title{Notes on Abbot's Introduction to Analysis}
\author{Zaid Khan}
\date{\today}



\begin{document}
\maketitle
\clearpage

\tableofcontents


\clearpage


\section{TODO}


\section{Sequences and Series}
\subsection{The Limit of  Sequence}
\subsubsection{Definition of convergence for a sequence}
A sequence converges to some number \textit{a} if \begin{math} \forall \epsilon > 0 \quad \exists N \in \mathbb{N} \end{math} such that \begin{math} \forall n \geq N \implies | a_{n} - a | < \epsilon     \end{math}.

\subsubsection{Outline of a Proof of Convergence}


\begin{itemize}

\item Let \( \epsilon > 0\) be arbitrary.

\item Demonstrate a choice of \( N \in \mathbb{N} \). This essentially means solve for N in terms of \( \epsilon > 0 \) .

\item Show that \textit{N} actually works. This is simply the inequality that you derived in the previous item.

\item Assume \( n \geq N \).

\item Derive the inequality \( | a_{n}  - a | < \epsilon    \) . Again, this is simply equivalent to solving for \( \epsilon \) in an inequality.


\end{itemize}

\subsection{The Monotone Convergence Theorem}

A sequence is \textbf{monotone} if it is either increasing or
decreasing.

\subsubsection{Monotone Convergence Theorem}
If a sequence is monotone and bounded, then it converges.

An infinite series is said to be converging if the sequence of partial
sums is converging. Note how the definition of convergence of a series
is linked back to the notion of convergence for a sequence. 

 
\subsection{Subsequences and the Bolzano-Weierstrass Theorem}

\subsubsection{Subsequence Convergence Theorem}
The subsequences of a convergent sequence converge to the same limit as the original sequence.


\subsubsection{Bolzano-Weierstrass Theorem}
There exists a convergent subsequence within every bounded sequence.



\subsection{The Cauchy Criterion}

\subsubsection{Cauchy Criterion Definition}

A sequence \( a_{n} \) is called a \textit{Cauchy sequence} if \( \forall \epsilon >0 \quad \exists \quad N \in \mathbb{N} : m,n \geq N \implies \left | a_{n} - a_{m}    \right |  < \epsilon  \) . To simplify, a sequence is cauchy if there exists a point in the sequence after which all the numbers are closer to each other than to any arbitrary epsilon.


\subsubsection{Cauchy Convergence}

\begin{itemize}

\item Every convergent sequence is a Cauchy sequence.

\item Cauchy sequences are bounded.

\item A sequence converges if and only if it is a Cauchy Sequence.

\end{itemize}


\subsubsection{Effects of Completeness on Convergence}
Completeness is a necessary condition in order for the above definitions of convergence to work correctly. In particular, we need a way to guarantee that when a sequence is converging, the number that it is converging to is actually there. This is part of the reason why the real numbers were introduced - the rationals have holes in them. The Axiom of Completeness, Nested Interval Property, Bolzano-Weierstrauss,Cauchy Criterion, and Monotone Convergence Theorem are all equivalent.


\subsection{Properties of Infinite Series}

\subsubsection{Convergence of an Infinite Series}
The convergence of an infinite series \( a_{k} \) is defined in terms of the sequence of its partial sums \( s_{n} \). \( \Sigma_{1}^{\infty} a_{k} = A \) means that \( lim(s_{n}) = A  \).


\subsubsection{Cauchy Criterion for Series}

\begin{itemize}

\item The series \(  \Sigma_{k=1}^{\infty} \) converges if and only if, given \( \epsilon > 0 \quad \exists N \in \mathbb{N} : n>m \geq N \implies \left | a_{m+1}+ a_{m+2} + \ldots + a_{n}     \right | < \epsilon \).

\item \textbf{Corrollary:} If the series \( \Sigma_{k=1}^{\infty} a_{k}  \) converges, then \( a_{k} \rightarrow 0 \).

\end{itemize}


\subsubsection{Comparison Test}
Assume \( (a_{k}) \) and \( (b_{k})  \) are sequences satisfying \( 0 \leq a_{k} \leq b_{k} \quad \forall k \in \mathbb{N} \).


\begin{itemize}

\item If \( \Sigma b_{k}  \) converges, then \(  \Sigma a_{k}  \) also converges.

\item If \( \Sigma a_{k} \) diverges, then \( \Sigma b_{k} \) also diverges.


\end{itemize}

\subsubsection{Absolute Convergence Test}

If the series \( \Sigma \left |  a_{k}  \right |  \) converges, then \( \Sigma a_{k}  \) converges as well. A sequence is said to converge absolutely if for a sequence \( a_{k} \) , \(  \sigma_{k=1}^{\infty} | a_{k} |  \) also converges. Otherwise, the sequence is said to converge conditionally.

\subsubsection{Alternating Series Test}
Let \( (a_{n}) \) be a sequence that is monotonically decreasing and has a limit of 0. Then, the alternating series \( \Sigma_{n=1}^{\infty} (-1)^{n+1} a_{n}   \) converges.

\subsubsection{Absolute Convergent Rearrangement Theorem For Series}

If a series converges absolutely, than any rearrangement of the series converges to the same limit.

\section{Topology of the Reals}


\subsection{Open and Closed Sets}

\subsubsection{Epsilon Neighborhoods}

An $ \epsilon -neighborhood $ is a set \( V_{\epsilon} (x) = \left \{ x \in \mathbb{R} : | x- a | < \epsilon \right \} \). Note that the interval is open - it does not contain the endpoints since the distance must be less than epsilon.

\subsubsection{Definition of an Open Set}

A set \( O \subseteq \mathbb{R} \) is \textit{open} if \( \forall a \in O \) there exists an $ \epsilon $ -neighborhood \( V_{\epsilon} (a) \subseteq O \). Recall that an $\epsilon - neighborhood$ is the set of points surrounding a point that are less than $ \epsilon $ away from it. This set particular set does not contain the endpoints. 

\subsubsection{Union and Intersection of Open Sets}

The union of an arbitrary collection of open sets is open. The intersection of a finite collection of open sets is open. 

\subsubsection{Limit Points}

A point $x$ is a \textit{limit point} of a set A if every $ \epsilon$-neighborhood  $V_{\epsilon}(x)$ of $x$ intersects A at some point other than A.  Alternate interpretation: a limit point is the limit of a sequence at the edge of the set.

\subsubsection{Limit Point Sequence Convergence}

A point $x$ is a limit point of a set $A$ if and only if $x=lim \quad a_n $ for some sequence $ (a_n) $ contained in $A$ satisfying \(a_n \neq x \quad \forall n \in A \).

\subsubsection{Definition of an Isolated Point}
A point $ a \in A $ is an isolated point of A if it is not a limit point of A.


\subsubsection{Definition of a Closed Set}
A set $ F \subseteq \mathbb{R} $ is closed if it contains its limit points.

\subsubsection{Closed Sets and Cauchy Sequences}
A set $ F \subseteq \mathbb{R} $ is closed if and only if every Cauchy sequence contained in F has a limit that is also an element of F.

\subsubsection{Density of Q in R}
For every $y \in \mathbb{R} $, there exists a sequence of rational numbers that converges to y.

\subsubsection{Closure}

Given a set $ A \subseteq \mathbb{R} $, let L be the set of all limit points of A. The set $A \cup L $ is defined to be the closure of $ \overline{A}$.

For any $ A \subseteq \mathbb{R} $, the closure $ \overline{A}$ is a closed set and the smallest closed set containing A.

\subsubsection{Complements and Open/Closed Sets}

A set O is open if an only if $ O^c$ is closed. LIkewise, a set F is closed if and only if $F^c$ is open.

\begin{itemize}

\item The union of a finite collection of finite sets is closed.

\item The intersection of an arbitrary collection of arbitrary sets is closed.

\end{itemize}

\subsection{Compact Sets}


\subsubsection{Compactness}

A set $ K \subseteq \mathbb{R} $ is compact if every sequence in K has a subsequence that converges to a limit which is also in K.

\subsubsection{Characterization of Compactness in R}
A set $ K \subseteq R$ is compact if and only if it is closed and bounded. A set $A$ is bounded if \( \exists M: |a| \leq M \forall a \in A     \).

\subsubsection{Nested Compact Set Property}

If \( K_1 \supseteq K_2 \supseteq K_3 \supseteq \ldots K_n    \) is a nested sequence of nonempty compact, sets, than the intersection \( \bigcap_{n=1}^{\infty} K_n      \) is not empty.

\subsubsection{Open Covers}

Let $ A \subseteq \mathbb{R} $. An open cover for A is a possibly infinite collection of open sets whose union contains $A$. A finite subcover is finite collection of open sets from the original open cover whose union still completely contains $A$.

\subsubsection{Heine-Borel Theorem}
Let K be a subset of $\mathbb{R}$. All of the following statements are equivalent in the sense that any one of them implies the two others.

\begin{itemize}

\item K is compact.

\item K is closed and bounded.

\item Every open cover for K has a finite subcover.

\end{itemize}


\subsection{Perfect Sets and Connected Sets}

\subsubsection{Perfect Sets}

A set $ P \subseteq \mathbb{R} $ is perfect if it is closed and contains no isolated points. 

\subsubsection{Cardinality of Perfect Sets}
Any nonempty perfect set is uncountable.

\subsubsection{Separated Sets}
Two nonempty sets $A,B \subseteq \mathbb{R} $ are separated if $ \overline{A} \cap B $ and $ \overline{B} \cap A$ are both empty.

\subsubsection{Disconnected and Connected Sets}
A set $E \subseteq \mathbb{R} $ is disconnected if it can be written as $ E = A \cup B $, where $A$ and $B$ are nonempty sepaarated sets. A set that is not disconnected is called a connected set. 

\subsubsection{Properties of Connected Sets}

\begin{itemize}

\item A set $ E \subseteq \mathbb{R} $ is connected if and only if, for all nonempty disjoint sets $A$ and $B$ satisfying $E= A \cup B$, there always exists a convergent sequence $(x_n) \rightarrow x$ with $(x_n)$ contained in one of $A$ or $B$, and x an element of the other.

\item A set $E \subseteq \mathbb{R}$ is connected if and only if whenever $ a < c< b $ with $ a, b \in E$, it follows that $c \ in E$ as well. 

\end{itemize}


\section{Functional Limits}




\end{document}